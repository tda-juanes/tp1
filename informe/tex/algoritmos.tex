\subsection{Algoritmo}

Para minimizar $\mathcal{M}$, proponemos un algoritmo \textit{Greedy} que siempre toma el elemento de mayor $a_i$. Esto es, ordenando de mayor a menor ignorando los valores de $s_i$.

\subsubsection{Optimalidad}

Sabemos que al intercambiar dos elementos, solo es necesario ver que pasa con $\mathcal{S}_i + a_i$ y $\mathcal{S}_{i+1} + a_{i+1}$, ya que $\mathcal{S}_k + a_k$ es independiente del orden en el que est\'an los elementos anteriores, o los que le siguen. Consideremos que pasa al intercambiar dos elementos adyacentes $d_i$ y $d_{i+1}$:

\begin{equation*}
    A := \mathcal{S}_i + a_i
\end{equation*}
\begin{equation*}
    B := \mathcal{S}_i + s_{i+1} + a_{i+1}
\end{equation*}

Al intercambiarlos, obtendr\'iamos:

\begin{equation*}
    A' := \mathcal{S}_i + s_{i+1} + a_i
\end{equation*}
\begin{equation*}
    B' := \mathcal{S}_i + a_{i+1}
\end{equation*}

\begin{itemize}
    \item $a_i < a_{i+1} \implies B > A' \land B > B'$. Intercambiamos, $\mathcal{M'} \le \mathcal{M}$.
    \item $a_i = a_{i+1} \implies B = A' \land A = B'$. Intercambiemos o no, $\mathcal{M'} = \mathcal{M}$.
    \item $a_i > a_{i+1} \implies A < B' \land B < B'$. No intercambiamos. $\mathcal{M'} = \mathcal{M}$.
\end{itemize}

Con esto en mente, si partimos de una configuraci\'on \'optima, podemos realizar intercambios hasta ordenar los elementos seg\'un nuestro criterio sin empeorar el valor de $\mathcal{M}$, lo que demuestra que nuestro criterio es tan bueno como el \'optimo.

\subsection{Implementaci\'on}

\begin{lstlisting}[language=Python]
from collections import namedtuple
import heapq

Demora = namedtuple("Demora", ['s', 'a'])
Demora.__le__ = lambda self, other: self.a >= other.a
Demora.__ge__ = lambda self, other: self.a <= other.a
Demora.__lt__ = lambda self, other: self.a > other.a
Demora.__gt__ = lambda self, other: self.a < other.a

def ordenar(datos: list[Demora]):
    """ Requiere que (x < y) <=> (x.a > y.a) """
    heapq.heapify(datos)
    return [heapq.heappop(datos) for _ in range(len(datos))]
\end{lstlisting}

La ecuaci\'on de recurrencia que corresponde a este algoritmo es: 
\begin{equation*} %nota: el asterisco es para que no aparezca el (1) al lado de la ecuación
    \mathcal{T}(n) = \mathcal{T}\left(n - 1\right) + \mathcal{O}\left(\log n\right)
\end{equation*}

Esto es porque la operaci\'on \texttt{heapify} es lineal, y \texttt{heappop} es logar\'itmica, al iterar por todos los elementos, obtenemos un ordenamiento $\mathcal{O}\left(n \log n\right)$.
